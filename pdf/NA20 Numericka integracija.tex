
% Default to the notebook output style

    


% Inherit from the specified cell style.


    
\documentclass[11pt]{article}

    
    
    \usepackage[T1]{fontenc}
    % Nicer default font (+ math font) than Computer Modern for most use cases
    \usepackage{mathpazo}

    % Basic figure setup, for now with no caption control since it's done
    % automatically by Pandoc (which extracts ![](path) syntax from Markdown).
    \usepackage{graphicx}
    % We will generate all images so they have a width \maxwidth. This means
    % that they will get their normal width if they fit onto the page, but
    % are scaled down if they would overflow the margins.
    \makeatletter
    \def\maxwidth{\ifdim\Gin@nat@width>\linewidth\linewidth
    \else\Gin@nat@width\fi}
    \makeatother
    \let\Oldincludegraphics\includegraphics
    % Set max figure width to be 80% of text width, for now hardcoded.
    \renewcommand{\includegraphics}[1]{\Oldincludegraphics[width=.8\maxwidth]{#1}}
    % Ensure that by default, figures have no caption (until we provide a
    % proper Figure object with a Caption API and a way to capture that
    % in the conversion process - todo).
    \usepackage{caption}
    \DeclareCaptionLabelFormat{nolabel}{}
    \captionsetup{labelformat=nolabel}

    \usepackage{adjustbox} % Used to constrain images to a maximum size 
    \usepackage{xcolor} % Allow colors to be defined
    \usepackage{enumerate} % Needed for markdown enumerations to work
    \usepackage{geometry} % Used to adjust the document margins
    \usepackage{amsmath} % Equations
    \usepackage{amssymb} % Equations
    \usepackage{textcomp} % defines textquotesingle
    % Hack from http://tex.stackexchange.com/a/47451/13684:
    \AtBeginDocument{%
        \def\PYZsq{\textquotesingle}% Upright quotes in Pygmentized code
    }
    \usepackage{upquote} % Upright quotes for verbatim code
    \usepackage{eurosym} % defines \euro
    \usepackage[mathletters]{ucs} % Extended unicode (utf-8) support
    \usepackage[utf8x]{inputenc} % Allow utf-8 characters in the tex document
    \usepackage{fancyvrb} % verbatim replacement that allows latex
    \usepackage{grffile} % extends the file name processing of package graphics 
                         % to support a larger range 
    % The hyperref package gives us a pdf with properly built
    % internal navigation ('pdf bookmarks' for the table of contents,
    % internal cross-reference links, web links for URLs, etc.)
    \usepackage{hyperref}
    \usepackage{longtable} % longtable support required by pandoc >1.10
    \usepackage{booktabs}  % table support for pandoc > 1.12.2
    \usepackage[inline]{enumitem} % IRkernel/repr support (it uses the enumerate* environment)
    \usepackage[normalem]{ulem} % ulem is needed to support strikethroughs (\sout)
                                % normalem makes italics be italics, not underlines
    

    
    
    % Colors for the hyperref package
    \definecolor{urlcolor}{rgb}{0,.145,.698}
    \definecolor{linkcolor}{rgb}{.71,0.21,0.01}
    \definecolor{citecolor}{rgb}{.12,.54,.11}

    % ANSI colors
    \definecolor{ansi-black}{HTML}{3E424D}
    \definecolor{ansi-black-intense}{HTML}{282C36}
    \definecolor{ansi-red}{HTML}{E75C58}
    \definecolor{ansi-red-intense}{HTML}{B22B31}
    \definecolor{ansi-green}{HTML}{00A250}
    \definecolor{ansi-green-intense}{HTML}{007427}
    \definecolor{ansi-yellow}{HTML}{DDB62B}
    \definecolor{ansi-yellow-intense}{HTML}{B27D12}
    \definecolor{ansi-blue}{HTML}{208FFB}
    \definecolor{ansi-blue-intense}{HTML}{0065CA}
    \definecolor{ansi-magenta}{HTML}{D160C4}
    \definecolor{ansi-magenta-intense}{HTML}{A03196}
    \definecolor{ansi-cyan}{HTML}{60C6C8}
    \definecolor{ansi-cyan-intense}{HTML}{258F8F}
    \definecolor{ansi-white}{HTML}{C5C1B4}
    \definecolor{ansi-white-intense}{HTML}{A1A6B2}

    % commands and environments needed by pandoc snippets
    % extracted from the output of `pandoc -s`
    \providecommand{\tightlist}{%
      \setlength{\itemsep}{0pt}\setlength{\parskip}{0pt}}
    \DefineVerbatimEnvironment{Highlighting}{Verbatim}{commandchars=\\\{\}}
    % Add ',fontsize=\small' for more characters per line
    \newenvironment{Shaded}{}{}
    \newcommand{\KeywordTok}[1]{\textcolor[rgb]{0.00,0.44,0.13}{\textbf{{#1}}}}
    \newcommand{\DataTypeTok}[1]{\textcolor[rgb]{0.56,0.13,0.00}{{#1}}}
    \newcommand{\DecValTok}[1]{\textcolor[rgb]{0.25,0.63,0.44}{{#1}}}
    \newcommand{\BaseNTok}[1]{\textcolor[rgb]{0.25,0.63,0.44}{{#1}}}
    \newcommand{\FloatTok}[1]{\textcolor[rgb]{0.25,0.63,0.44}{{#1}}}
    \newcommand{\CharTok}[1]{\textcolor[rgb]{0.25,0.44,0.63}{{#1}}}
    \newcommand{\StringTok}[1]{\textcolor[rgb]{0.25,0.44,0.63}{{#1}}}
    \newcommand{\CommentTok}[1]{\textcolor[rgb]{0.38,0.63,0.69}{\textit{{#1}}}}
    \newcommand{\OtherTok}[1]{\textcolor[rgb]{0.00,0.44,0.13}{{#1}}}
    \newcommand{\AlertTok}[1]{\textcolor[rgb]{1.00,0.00,0.00}{\textbf{{#1}}}}
    \newcommand{\FunctionTok}[1]{\textcolor[rgb]{0.02,0.16,0.49}{{#1}}}
    \newcommand{\RegionMarkerTok}[1]{{#1}}
    \newcommand{\ErrorTok}[1]{\textcolor[rgb]{1.00,0.00,0.00}{\textbf{{#1}}}}
    \newcommand{\NormalTok}[1]{{#1}}
    
    % Additional commands for more recent versions of Pandoc
    \newcommand{\ConstantTok}[1]{\textcolor[rgb]{0.53,0.00,0.00}{{#1}}}
    \newcommand{\SpecialCharTok}[1]{\textcolor[rgb]{0.25,0.44,0.63}{{#1}}}
    \newcommand{\VerbatimStringTok}[1]{\textcolor[rgb]{0.25,0.44,0.63}{{#1}}}
    \newcommand{\SpecialStringTok}[1]{\textcolor[rgb]{0.73,0.40,0.53}{{#1}}}
    \newcommand{\ImportTok}[1]{{#1}}
    \newcommand{\DocumentationTok}[1]{\textcolor[rgb]{0.73,0.13,0.13}{\textit{{#1}}}}
    \newcommand{\AnnotationTok}[1]{\textcolor[rgb]{0.38,0.63,0.69}{\textbf{\textit{{#1}}}}}
    \newcommand{\CommentVarTok}[1]{\textcolor[rgb]{0.38,0.63,0.69}{\textbf{\textit{{#1}}}}}
    \newcommand{\VariableTok}[1]{\textcolor[rgb]{0.10,0.09,0.49}{{#1}}}
    \newcommand{\ControlFlowTok}[1]{\textcolor[rgb]{0.00,0.44,0.13}{\textbf{{#1}}}}
    \newcommand{\OperatorTok}[1]{\textcolor[rgb]{0.40,0.40,0.40}{{#1}}}
    \newcommand{\BuiltInTok}[1]{{#1}}
    \newcommand{\ExtensionTok}[1]{{#1}}
    \newcommand{\PreprocessorTok}[1]{\textcolor[rgb]{0.74,0.48,0.00}{{#1}}}
    \newcommand{\AttributeTok}[1]{\textcolor[rgb]{0.49,0.56,0.16}{{#1}}}
    \newcommand{\InformationTok}[1]{\textcolor[rgb]{0.38,0.63,0.69}{\textbf{\textit{{#1}}}}}
    \newcommand{\WarningTok}[1]{\textcolor[rgb]{0.38,0.63,0.69}{\textbf{\textit{{#1}}}}}
    
    
    % Define a nice break command that doesn't care if a line doesn't already
    % exist.
    \def\br{\hspace*{\fill} \\* }
    % Math Jax compatability definitions
    \def\gt{>}
    \def\lt{<}
    % Document parameters
    \title{NA20 Numericka integracija}
    
    
    

    % Pygments definitions
    
\makeatletter
\def\PY@reset{\let\PY@it=\relax \let\PY@bf=\relax%
    \let\PY@ul=\relax \let\PY@tc=\relax%
    \let\PY@bc=\relax \let\PY@ff=\relax}
\def\PY@tok#1{\csname PY@tok@#1\endcsname}
\def\PY@toks#1+{\ifx\relax#1\empty\else%
    \PY@tok{#1}\expandafter\PY@toks\fi}
\def\PY@do#1{\PY@bc{\PY@tc{\PY@ul{%
    \PY@it{\PY@bf{\PY@ff{#1}}}}}}}
\def\PY#1#2{\PY@reset\PY@toks#1+\relax+\PY@do{#2}}

\expandafter\def\csname PY@tok@w\endcsname{\def\PY@tc##1{\textcolor[rgb]{0.73,0.73,0.73}{##1}}}
\expandafter\def\csname PY@tok@c\endcsname{\let\PY@it=\textit\def\PY@tc##1{\textcolor[rgb]{0.25,0.50,0.50}{##1}}}
\expandafter\def\csname PY@tok@cp\endcsname{\def\PY@tc##1{\textcolor[rgb]{0.74,0.48,0.00}{##1}}}
\expandafter\def\csname PY@tok@k\endcsname{\let\PY@bf=\textbf\def\PY@tc##1{\textcolor[rgb]{0.00,0.50,0.00}{##1}}}
\expandafter\def\csname PY@tok@kp\endcsname{\def\PY@tc##1{\textcolor[rgb]{0.00,0.50,0.00}{##1}}}
\expandafter\def\csname PY@tok@kt\endcsname{\def\PY@tc##1{\textcolor[rgb]{0.69,0.00,0.25}{##1}}}
\expandafter\def\csname PY@tok@o\endcsname{\def\PY@tc##1{\textcolor[rgb]{0.40,0.40,0.40}{##1}}}
\expandafter\def\csname PY@tok@ow\endcsname{\let\PY@bf=\textbf\def\PY@tc##1{\textcolor[rgb]{0.67,0.13,1.00}{##1}}}
\expandafter\def\csname PY@tok@nb\endcsname{\def\PY@tc##1{\textcolor[rgb]{0.00,0.50,0.00}{##1}}}
\expandafter\def\csname PY@tok@nf\endcsname{\def\PY@tc##1{\textcolor[rgb]{0.00,0.00,1.00}{##1}}}
\expandafter\def\csname PY@tok@nc\endcsname{\let\PY@bf=\textbf\def\PY@tc##1{\textcolor[rgb]{0.00,0.00,1.00}{##1}}}
\expandafter\def\csname PY@tok@nn\endcsname{\let\PY@bf=\textbf\def\PY@tc##1{\textcolor[rgb]{0.00,0.00,1.00}{##1}}}
\expandafter\def\csname PY@tok@ne\endcsname{\let\PY@bf=\textbf\def\PY@tc##1{\textcolor[rgb]{0.82,0.25,0.23}{##1}}}
\expandafter\def\csname PY@tok@nv\endcsname{\def\PY@tc##1{\textcolor[rgb]{0.10,0.09,0.49}{##1}}}
\expandafter\def\csname PY@tok@no\endcsname{\def\PY@tc##1{\textcolor[rgb]{0.53,0.00,0.00}{##1}}}
\expandafter\def\csname PY@tok@nl\endcsname{\def\PY@tc##1{\textcolor[rgb]{0.63,0.63,0.00}{##1}}}
\expandafter\def\csname PY@tok@ni\endcsname{\let\PY@bf=\textbf\def\PY@tc##1{\textcolor[rgb]{0.60,0.60,0.60}{##1}}}
\expandafter\def\csname PY@tok@na\endcsname{\def\PY@tc##1{\textcolor[rgb]{0.49,0.56,0.16}{##1}}}
\expandafter\def\csname PY@tok@nt\endcsname{\let\PY@bf=\textbf\def\PY@tc##1{\textcolor[rgb]{0.00,0.50,0.00}{##1}}}
\expandafter\def\csname PY@tok@nd\endcsname{\def\PY@tc##1{\textcolor[rgb]{0.67,0.13,1.00}{##1}}}
\expandafter\def\csname PY@tok@s\endcsname{\def\PY@tc##1{\textcolor[rgb]{0.73,0.13,0.13}{##1}}}
\expandafter\def\csname PY@tok@sd\endcsname{\let\PY@it=\textit\def\PY@tc##1{\textcolor[rgb]{0.73,0.13,0.13}{##1}}}
\expandafter\def\csname PY@tok@si\endcsname{\let\PY@bf=\textbf\def\PY@tc##1{\textcolor[rgb]{0.73,0.40,0.53}{##1}}}
\expandafter\def\csname PY@tok@se\endcsname{\let\PY@bf=\textbf\def\PY@tc##1{\textcolor[rgb]{0.73,0.40,0.13}{##1}}}
\expandafter\def\csname PY@tok@sr\endcsname{\def\PY@tc##1{\textcolor[rgb]{0.73,0.40,0.53}{##1}}}
\expandafter\def\csname PY@tok@ss\endcsname{\def\PY@tc##1{\textcolor[rgb]{0.10,0.09,0.49}{##1}}}
\expandafter\def\csname PY@tok@sx\endcsname{\def\PY@tc##1{\textcolor[rgb]{0.00,0.50,0.00}{##1}}}
\expandafter\def\csname PY@tok@m\endcsname{\def\PY@tc##1{\textcolor[rgb]{0.40,0.40,0.40}{##1}}}
\expandafter\def\csname PY@tok@gh\endcsname{\let\PY@bf=\textbf\def\PY@tc##1{\textcolor[rgb]{0.00,0.00,0.50}{##1}}}
\expandafter\def\csname PY@tok@gu\endcsname{\let\PY@bf=\textbf\def\PY@tc##1{\textcolor[rgb]{0.50,0.00,0.50}{##1}}}
\expandafter\def\csname PY@tok@gd\endcsname{\def\PY@tc##1{\textcolor[rgb]{0.63,0.00,0.00}{##1}}}
\expandafter\def\csname PY@tok@gi\endcsname{\def\PY@tc##1{\textcolor[rgb]{0.00,0.63,0.00}{##1}}}
\expandafter\def\csname PY@tok@gr\endcsname{\def\PY@tc##1{\textcolor[rgb]{1.00,0.00,0.00}{##1}}}
\expandafter\def\csname PY@tok@ge\endcsname{\let\PY@it=\textit}
\expandafter\def\csname PY@tok@gs\endcsname{\let\PY@bf=\textbf}
\expandafter\def\csname PY@tok@gp\endcsname{\let\PY@bf=\textbf\def\PY@tc##1{\textcolor[rgb]{0.00,0.00,0.50}{##1}}}
\expandafter\def\csname PY@tok@go\endcsname{\def\PY@tc##1{\textcolor[rgb]{0.53,0.53,0.53}{##1}}}
\expandafter\def\csname PY@tok@gt\endcsname{\def\PY@tc##1{\textcolor[rgb]{0.00,0.27,0.87}{##1}}}
\expandafter\def\csname PY@tok@err\endcsname{\def\PY@bc##1{\setlength{\fboxsep}{0pt}\fcolorbox[rgb]{1.00,0.00,0.00}{1,1,1}{\strut ##1}}}
\expandafter\def\csname PY@tok@kc\endcsname{\let\PY@bf=\textbf\def\PY@tc##1{\textcolor[rgb]{0.00,0.50,0.00}{##1}}}
\expandafter\def\csname PY@tok@kd\endcsname{\let\PY@bf=\textbf\def\PY@tc##1{\textcolor[rgb]{0.00,0.50,0.00}{##1}}}
\expandafter\def\csname PY@tok@kn\endcsname{\let\PY@bf=\textbf\def\PY@tc##1{\textcolor[rgb]{0.00,0.50,0.00}{##1}}}
\expandafter\def\csname PY@tok@kr\endcsname{\let\PY@bf=\textbf\def\PY@tc##1{\textcolor[rgb]{0.00,0.50,0.00}{##1}}}
\expandafter\def\csname PY@tok@bp\endcsname{\def\PY@tc##1{\textcolor[rgb]{0.00,0.50,0.00}{##1}}}
\expandafter\def\csname PY@tok@fm\endcsname{\def\PY@tc##1{\textcolor[rgb]{0.00,0.00,1.00}{##1}}}
\expandafter\def\csname PY@tok@vc\endcsname{\def\PY@tc##1{\textcolor[rgb]{0.10,0.09,0.49}{##1}}}
\expandafter\def\csname PY@tok@vg\endcsname{\def\PY@tc##1{\textcolor[rgb]{0.10,0.09,0.49}{##1}}}
\expandafter\def\csname PY@tok@vi\endcsname{\def\PY@tc##1{\textcolor[rgb]{0.10,0.09,0.49}{##1}}}
\expandafter\def\csname PY@tok@vm\endcsname{\def\PY@tc##1{\textcolor[rgb]{0.10,0.09,0.49}{##1}}}
\expandafter\def\csname PY@tok@sa\endcsname{\def\PY@tc##1{\textcolor[rgb]{0.73,0.13,0.13}{##1}}}
\expandafter\def\csname PY@tok@sb\endcsname{\def\PY@tc##1{\textcolor[rgb]{0.73,0.13,0.13}{##1}}}
\expandafter\def\csname PY@tok@sc\endcsname{\def\PY@tc##1{\textcolor[rgb]{0.73,0.13,0.13}{##1}}}
\expandafter\def\csname PY@tok@dl\endcsname{\def\PY@tc##1{\textcolor[rgb]{0.73,0.13,0.13}{##1}}}
\expandafter\def\csname PY@tok@s2\endcsname{\def\PY@tc##1{\textcolor[rgb]{0.73,0.13,0.13}{##1}}}
\expandafter\def\csname PY@tok@sh\endcsname{\def\PY@tc##1{\textcolor[rgb]{0.73,0.13,0.13}{##1}}}
\expandafter\def\csname PY@tok@s1\endcsname{\def\PY@tc##1{\textcolor[rgb]{0.73,0.13,0.13}{##1}}}
\expandafter\def\csname PY@tok@mb\endcsname{\def\PY@tc##1{\textcolor[rgb]{0.40,0.40,0.40}{##1}}}
\expandafter\def\csname PY@tok@mf\endcsname{\def\PY@tc##1{\textcolor[rgb]{0.40,0.40,0.40}{##1}}}
\expandafter\def\csname PY@tok@mh\endcsname{\def\PY@tc##1{\textcolor[rgb]{0.40,0.40,0.40}{##1}}}
\expandafter\def\csname PY@tok@mi\endcsname{\def\PY@tc##1{\textcolor[rgb]{0.40,0.40,0.40}{##1}}}
\expandafter\def\csname PY@tok@il\endcsname{\def\PY@tc##1{\textcolor[rgb]{0.40,0.40,0.40}{##1}}}
\expandafter\def\csname PY@tok@mo\endcsname{\def\PY@tc##1{\textcolor[rgb]{0.40,0.40,0.40}{##1}}}
\expandafter\def\csname PY@tok@ch\endcsname{\let\PY@it=\textit\def\PY@tc##1{\textcolor[rgb]{0.25,0.50,0.50}{##1}}}
\expandafter\def\csname PY@tok@cm\endcsname{\let\PY@it=\textit\def\PY@tc##1{\textcolor[rgb]{0.25,0.50,0.50}{##1}}}
\expandafter\def\csname PY@tok@cpf\endcsname{\let\PY@it=\textit\def\PY@tc##1{\textcolor[rgb]{0.25,0.50,0.50}{##1}}}
\expandafter\def\csname PY@tok@c1\endcsname{\let\PY@it=\textit\def\PY@tc##1{\textcolor[rgb]{0.25,0.50,0.50}{##1}}}
\expandafter\def\csname PY@tok@cs\endcsname{\let\PY@it=\textit\def\PY@tc##1{\textcolor[rgb]{0.25,0.50,0.50}{##1}}}

\def\PYZbs{\char`\\}
\def\PYZus{\char`\_}
\def\PYZob{\char`\{}
\def\PYZcb{\char`\}}
\def\PYZca{\char`\^}
\def\PYZam{\char`\&}
\def\PYZlt{\char`\<}
\def\PYZgt{\char`\>}
\def\PYZsh{\char`\#}
\def\PYZpc{\char`\%}
\def\PYZdl{\char`\$}
\def\PYZhy{\char`\-}
\def\PYZsq{\char`\'}
\def\PYZdq{\char`\"}
\def\PYZti{\char`\~}
% for compatibility with earlier versions
\def\PYZat{@}
\def\PYZlb{[}
\def\PYZrb{]}
\makeatother


    % Exact colors from NB
    \definecolor{incolor}{rgb}{0.0, 0.0, 0.5}
    \definecolor{outcolor}{rgb}{0.545, 0.0, 0.0}



    
    % Prevent overflowing lines due to hard-to-break entities
    \sloppy 
    % Setup hyperref package
    \hypersetup{
      breaklinks=true,  % so long urls are correctly broken across lines
      colorlinks=true,
      urlcolor=urlcolor,
      linkcolor=linkcolor,
      citecolor=citecolor,
      }
    % Slightly bigger margins than the latex defaults
    
    \geometry{verbose,tmargin=1in,bmargin=1in,lmargin=1in,rmargin=1in}
    
\author{Ivan Slapničar}
\date{4. siječnja 2019.}
\parindent=0pt
\parskip=1ex   

    \begin{document}
    
    
    \maketitle
    
    

    
    \hypertarget{numeriux10dka-integracija}{%
\section{Numerička integracija}\label{numeriux10dka-integracija}}

\hypertarget{newton-cotesove-formula}{%
\subsection{Newton-Cotesove formula}\label{newton-cotesove-formula}}

Funkcija \(f(x):[a,b]\to\mathbb{R}\) se interpolira polinomom stupnja
\(n\) u kroz \(n+1\) ravnomjerno raspoređenih točaka te se integral
aproksimira integralom interpolacijskog polinoma. Polinom možemo
računati u Lagrangeovom obliku\\
(vidi bilježnicu
\href{NA09\%20Interpolacijski\%20polinomi.ipynb}{NA09 Interpolacijski polinomi.ipynb}):

\[
L_k(x)=\prod_{{i=0}\atop {i\neq k}}^n \frac{x-x_i}{x_k-x_i}
\]

Tada je

\[
f(x)\approx P_n(x)=\sum_{k=0}^n f(x_k) L_k(x),
\]

pa je

\[
\int_a^b f(x)\, dx\approx \int_a^b P_n(x) \, dx=\sum_{k=0}^n f(x_k) \int_a^b L_k(x)\, dx =(b-a)\sum_{k=0}^n \omega_k f(x_k). \tag{1}
\]

Uz supstituciju \(x=a+(b-a)\,t\), \emph{težine} \(\omega_k\) su

\[
\omega_k=\frac{1}{b-a}\int_a^b L_k(x)\, dx = \int_0^1 \prod_{{i=0}\atop {i\neq k}}^n \frac{nt-i}{k-i} \, dt.
\]

\hypertarget{trapezna-formula}{%
\subsubsection{Trapezna formula}\label{trapezna-formula}}

Za \(n=1\) Newton-Cotesova formula (1) daje

\[\omega_0=\omega_1=\frac{1}{2}.\]

Interval \([a,b]\) podjelimo na \(n\) jednakih podintervala,

\[[x_{i-1},x_{i}],\quad  i=1,2,\ldots,n,\]

i uvedimo oznake

\[ 
\Delta x=\frac{b-a}{n}, \quad y_i=f(x_i).
\]

Primijena Newton-Cotesove formule na svaki podinterval i zbrajanje daje
\emph{trapeznu formulu}:

\[
I_n=\Delta x\bigg( \frac{y_0}{2} +y_1+y_2+\cdots +y_{n-1}+\frac{y_n}{2}\bigg).
\]

Vrijedi

\[
\int_a^b f(x)\, dx=I_n+R,
\]

pri čemu je \emph{pogreška} \(R\) omeđena s

\[
|R|\leq \frac{b-a}{12}(\Delta x)^2 \max_{x\in(a,b)} |f''(x)|.
\]

Izvod trapezne formule i ocjene pogreške dan je u knjigama
\href{http://www.mathos.unios.hr/pim/Materijali/Num.pdf}{Numerička
matematika, poglavlje 7.1} i
\href{http://lavica.fesb.hr/mat2/predavanja/node46.html}{Matematika 2,
poglavlje 2.7.2}.

\hypertarget{simpsonova-formula}{%
\subsubsection{Simpsonova formula}\label{simpsonova-formula}}

Za \(n=2\) Newton-Cotesova formula (1) daje

\[\omega_0=\frac{1}{6},\quad \omega_1=\frac{2}{3},\quad \omega_2=\frac{1}{6}.\]

Interval \([a,b]\) podjelimo na paran broj \(n\) jednakih podintervala,
na svaki podinterval

\[[x_{2i-1},x_{2i+1}],\quad i=1,2,\ldots,\frac{n}{2},\]

primijenimo Newton-Cotesovu formulu i zbrojimo, što daje
\emph{Simpsonovu formulu}:

\[
I_n=\frac{\Delta x}{3}\big(y_0 +4(y_1+y_3\cdots +y_{n-1})+2(y_2+y_4+\cdots+y_{n-2})+y_n\big).
\]

Vrijedi

\[
\int_a^b f(x)\, dx =I_n+R,
\]

pri čemu je \emph{pogreška} \(R\) omeđena s

\[
|R|\leq \frac{b-a}{180}(\Delta x)^4 \max_{x\in(a,b)} |f^{(4)}(x)|. \tag{2}
\]

Za detalje vidi knjige
\href{http://www.mathos.unios.hr/pim/Materijali/Num.pdf}{Numerička
matematika, poglavlje 7.3} i
\href{http://lavica.fesb.hr/mat2/predavanja/node46.html}{Matematika 2,
poglavlje 2.7.3}.

\hypertarget{richardson-ova-ekstrapolacija}{%
\subsubsection{Richardsonova
ekstrapolacija}\label{richardson-ova-ekstrapolacija}}

Ocjena pogreške pomoću formula (2) i (3) može biti složena.
\emph{Richardsonova ekstrapolacija} nam omogućuje da, uz određene
uvjete, pogrešku procijenimo koristeći aproksimaciju integrala s \(n/2\)
točaka. Ako se u ocjeni pogreške javlja član \((\Delta x)^m\) (\(m=2\)
za trapeznu formulu i \(m=4\) za Simpsonovu formulu), tada je pogreška
približno manja od broja (vidi
\href{http://lavica.fesb.hr/mat2/predavanja/node46.html}{Matematika 2,
poglavlje 2.7.4})

\[
E=\frac{\big(\frac{n}{2}\big)^m}{n^m-\big(\frac{n}{2}\big)^m}(I_n-I_{n/2}).
\]

Predznak broja \(E\) daje i predznak pogreške, ondosno, ako je \(E>0\),
tada je približno

\[\int_a^b f(x)\, dx\in[I_n,I_n+E],\]

a ako je \(E\leq 0\), tada je približno

\[\int_a^b f(x)\, dx\in[I_n+E,I_n].\]

    \begin{Verbatim}[commandchars=\\\{\}]
{\color{incolor}In [{\color{incolor}1}]:} \PY{k}{function} \PY{n}{mytrapez}\PY{p}{(}\PY{n}{f}\PY{o}{::}\PY{k+kt}{Function}\PY{p}{,}\PY{n}{a}\PY{o}{::}\PY{k+kt}{Number}\PY{p}{,}\PY{n}{b}\PY{o}{::}\PY{k+kt}{Number}\PY{p}{,}\PY{n}{n}\PY{o}{::}\PY{k+kt}{Int64}\PY{p}{)}
            \PY{c}{\PYZsh{} n je broj intervala}
            \PY{n}{X}\PY{o}{=}\PY{n}{range}\PY{p}{(}\PY{n}{a}\PY{p}{,}\PY{n}{stop}\PY{o}{=}\PY{n}{b}\PY{p}{,}\PY{n}{length}\PY{o}{=}\PY{n}{n}\PY{o}{+}\PY{l+m+mi}{1}\PY{p}{)}
            \PY{n}{Y}\PY{o}{=}\PY{n}{map}\PY{p}{(}\PY{n}{f}\PY{p}{,}\PY{n}{X}\PY{p}{)}
            \PY{n}{Δx}\PY{o}{=}\PY{p}{(}\PY{n}{b}\PY{o}{\PYZhy{}}\PY{n}{a}\PY{p}{)}\PY{o}{/}\PY{n}{n}
            \PY{n+nb}{I}\PY{o}{=}\PY{n}{Δx}\PY{o}{*}\PY{p}{(}\PY{n}{Y}\PY{p}{[}\PY{l+m+mi}{1}\PY{p}{]}\PY{o}{/}\PY{l+m+mi}{2}\PY{o}{+}\PY{n}{sum}\PY{p}{(}\PY{n}{Y}\PY{p}{[}\PY{l+m+mi}{2}\PY{o}{:}\PY{k}{end}\PY{o}{\PYZhy{}}\PY{l+m+mi}{1}\PY{p}{]}\PY{p}{)}\PY{o}{+}\PY{n}{Y}\PY{p}{[}\PY{k}{end}\PY{p}{]}\PY{o}{/}\PY{l+m+mi}{2}\PY{p}{)}
            \PY{c}{\PYZsh{} Richardsonova ekstrapolacija}
            \PY{n}{Ihalf}\PY{o}{=}\PY{l+m+mi}{2}\PY{o}{*}\PY{n}{Δx}\PY{o}{*}\PY{p}{(}\PY{n}{Y}\PY{p}{[}\PY{l+m+mi}{1}\PY{p}{]}\PY{o}{/}\PY{l+m+mi}{2}\PY{o}{+}\PY{n}{sum}\PY{p}{(}\PY{n}{Y}\PY{p}{[}\PY{l+m+mi}{3}\PY{o}{:}\PY{l+m+mi}{2}\PY{o}{:}\PY{k}{end}\PY{o}{\PYZhy{}}\PY{l+m+mi}{2}\PY{p}{]}\PY{p}{)}\PY{o}{+}\PY{n}{Y}\PY{p}{[}\PY{k}{end}\PY{p}{]}\PY{o}{/}\PY{l+m+mi}{2}\PY{p}{)}
            \PY{n}{E}\PY{o}{=}\PY{p}{(}\PY{n}{n}\PY{o}{/}\PY{l+m+mi}{2}\PY{p}{)}\PY{o}{\PYZca{}}\PY{l+m+mi}{2}\PY{o}{*}\PY{p}{(}\PY{n+nb}{I}\PY{o}{\PYZhy{}}\PY{n}{Ihalf}\PY{p}{)}\PY{o}{/}\PY{p}{(}\PY{n}{n}\PY{o}{\PYZca{}}\PY{l+m+mi}{2}\PY{o}{\PYZhy{}}\PY{p}{(}\PY{n}{n}\PY{o}{/}\PY{l+m+mi}{2}\PY{p}{)}\PY{o}{\PYZca{}}\PY{l+m+mi}{2}\PY{p}{)}
            \PY{n+nb}{I}\PY{p}{,}\PY{n}{E}
        \PY{k}{end}
\end{Verbatim}


\begin{Verbatim}[commandchars=\\\{\}]
{\color{outcolor}Out[{\color{outcolor}1}]:} mytrapez (generic function with 1 method)
\end{Verbatim}
            
    \begin{Verbatim}[commandchars=\\\{\}]
{\color{incolor}In [{\color{incolor}2}]:} \PY{k}{function} \PY{n}{mySimpson}\PY{p}{(}\PY{n}{f}\PY{o}{::}\PY{k+kt}{Function}\PY{p}{,}\PY{n}{a}\PY{o}{::}\PY{k+kt}{Number}\PY{p}{,}\PY{n}{b}\PY{o}{::}\PY{k+kt}{Number}\PY{p}{,}\PY{n}{n}\PY{o}{::}\PY{k+kt}{Int64}\PY{p}{)}
            \PY{c}{\PYZsh{} n je broj intervala, djeljiv s 4}
            \PY{n}{X}\PY{o}{=}\PY{n}{range}\PY{p}{(}\PY{n}{a}\PY{p}{,}\PY{n}{stop}\PY{o}{=}\PY{n}{b}\PY{p}{,}\PY{n}{length}\PY{o}{=}\PY{n}{n}\PY{o}{+}\PY{l+m+mi}{1}\PY{p}{)}
            \PY{n}{Y}\PY{o}{=}\PY{n}{map}\PY{p}{(}\PY{n}{f}\PY{p}{,}\PY{n}{X}\PY{p}{)}
            \PY{n}{Δx}\PY{o}{=}\PY{p}{(}\PY{n}{b}\PY{o}{\PYZhy{}}\PY{n}{a}\PY{p}{)}\PY{o}{/}\PY{n}{n}
            \PY{n+nb}{I}\PY{o}{=}\PY{n}{Δx}\PY{o}{/}\PY{l+m+mi}{3}\PY{o}{*}\PY{p}{(}\PY{n}{Y}\PY{p}{[}\PY{l+m+mi}{1}\PY{p}{]}\PY{o}{+}\PY{l+m+mi}{4}\PY{o}{*}\PY{n}{sum}\PY{p}{(}\PY{n}{Y}\PY{p}{[}\PY{l+m+mi}{2}\PY{o}{:}\PY{l+m+mi}{2}\PY{o}{:}\PY{k}{end}\PY{o}{\PYZhy{}}\PY{l+m+mi}{1}\PY{p}{]}\PY{p}{)}\PY{o}{+}\PY{l+m+mi}{2}\PY{o}{*}\PY{n}{sum}\PY{p}{(}\PY{n}{Y}\PY{p}{[}\PY{l+m+mi}{3}\PY{o}{:}\PY{l+m+mi}{2}\PY{o}{:}\PY{k}{end}\PY{o}{\PYZhy{}}\PY{l+m+mi}{2}\PY{p}{]}\PY{p}{)}\PY{o}{+}\PY{n}{Y}\PY{p}{[}\PY{k}{end}\PY{p}{]}\PY{p}{)}
            \PY{c}{\PYZsh{} Richardsonova ekstrapolacija}
            \PY{n}{Ihalf}\PY{o}{=}\PY{l+m+mi}{2}\PY{o}{*}\PY{n}{Δx}\PY{o}{/}\PY{l+m+mi}{3}\PY{o}{*}\PY{p}{(}\PY{n}{Y}\PY{p}{[}\PY{l+m+mi}{1}\PY{p}{]}\PY{o}{+}\PY{l+m+mi}{4}\PY{o}{*}\PY{n}{sum}\PY{p}{(}\PY{n}{Y}\PY{p}{[}\PY{l+m+mi}{3}\PY{o}{:}\PY{l+m+mi}{4}\PY{o}{:}\PY{k}{end}\PY{o}{\PYZhy{}}\PY{l+m+mi}{2}\PY{p}{]}\PY{p}{)}\PY{o}{+}\PY{l+m+mi}{2}\PY{o}{*}\PY{n}{sum}\PY{p}{(}\PY{n}{Y}\PY{p}{[}\PY{l+m+mi}{5}\PY{o}{:}\PY{l+m+mi}{4}\PY{o}{:}\PY{k}{end}\PY{o}{\PYZhy{}}\PY{l+m+mi}{4}\PY{p}{]}\PY{p}{)}\PY{o}{+}\PY{n}{Y}\PY{p}{[}\PY{k}{end}\PY{p}{]}\PY{p}{)}
            \PY{n}{E}\PY{o}{=}\PY{p}{(}\PY{n}{n}\PY{o}{/}\PY{l+m+mi}{2}\PY{p}{)}\PY{o}{\PYZca{}}\PY{l+m+mi}{4}\PY{o}{*}\PY{p}{(}\PY{n+nb}{I}\PY{o}{\PYZhy{}}\PY{n}{Ihalf}\PY{p}{)}\PY{o}{/}\PY{p}{(}\PY{n}{n}\PY{o}{\PYZca{}}\PY{l+m+mi}{4}\PY{o}{\PYZhy{}}\PY{p}{(}\PY{n}{n}\PY{o}{/}\PY{l+m+mi}{2}\PY{p}{)}\PY{o}{\PYZca{}}\PY{l+m+mi}{4}\PY{p}{)}
            \PY{n+nb}{I}\PY{p}{,}\PY{n}{E}
        \PY{k}{end}
\end{Verbatim}


\begin{Verbatim}[commandchars=\\\{\}]
{\color{outcolor}Out[{\color{outcolor}2}]:} mySimpson (generic function with 1 method)
\end{Verbatim}
            
    \hypertarget{primjer-1---eliptiux10dki-integral}{%
\subsubsection{Primjer 1 - Eliptički
integral}\label{primjer-1---eliptiux10dki-integral}}

Izračunajmo opseg elipse s polu-osima \(2\) i \(1\) (vidi
\href{http://lavica.fesb.hr/mat2/predavanja/node46.html}{Matematika 2,
poglavlje 2.7.1}). Elipsa je parametarski zadana s

\[
x=2\cos t,\quad y=\sin t,\quad t\in[0,\pi/2]
\]

pa je četvrtina opsega jednaka

\[
\frac{O}{4}\int\limits_0^{\pi/2} \sqrt{(-2\sin t)^2+(\cos t)^2}\, dt
=2\int\limits_0^{\pi/2} \sqrt{1-\frac{3}{4}(\cos t)^2}\, dt.
\]

Radi se o eliptičkom integralu druge vrste koji nije elementarno rješiv,
ali se može naći u
\href{http://nvlpubs.nist.gov/nistpubs/jres/50/jresv50n1p43_A1b.pdf}{tablicama}.

Vidimo da je \(O\approx 8\cdot 1.21125\).

    \begin{Verbatim}[commandchars=\\\{\}]
{\color{incolor}In [{\color{incolor}3}]:} \PY{n}{f1}\PY{p}{(}\PY{n}{x}\PY{p}{)}\PY{o}{=}\PY{n}{sqrt}\PY{p}{(}\PY{l+m+mi}{1}\PY{o}{\PYZhy{}}\PY{p}{(}\PY{l+m+mf}{3.0}\PY{p}{)}\PY{o}{/}\PY{l+m+mi}{4}\PY{o}{*}\PY{n}{cos}\PY{p}{(}\PY{n}{x}\PY{p}{)}\PY{o}{\PYZca{}}\PY{l+m+mi}{2}\PY{p}{)}
        \PY{n}{mytrapez}\PY{p}{(}\PY{n}{f1}\PY{p}{,}\PY{l+m+mi}{0}\PY{p}{,}\PY{n+nb}{π}\PY{o}{/}\PY{l+m+mi}{2}\PY{p}{,}\PY{l+m+mi}{4}\PY{p}{)}
\end{Verbatim}


\begin{Verbatim}[commandchars=\\\{\}]
{\color{outcolor}Out[{\color{outcolor}3}]:} (1.2110515487742433, 0.00036371987130023875)
\end{Verbatim}
            
    \begin{Verbatim}[commandchars=\\\{\}]
{\color{incolor}In [{\color{incolor}4}]:} \PY{n}{mytrapez}\PY{p}{(}\PY{n}{f1}\PY{p}{,}\PY{l+m+mi}{0}\PY{p}{,}\PY{n+nb}{pi}\PY{o}{/}\PY{l+m+mi}{2}\PY{p}{,}\PY{l+m+mi}{10}\PY{p}{)}
\end{Verbatim}


\begin{Verbatim}[commandchars=\\\{\}]
{\color{outcolor}Out[{\color{outcolor}4}]:} (1.2110560275664024, 1.172757710943273e-7)
\end{Verbatim}
            
    \begin{Verbatim}[commandchars=\\\{\}]
{\color{incolor}In [{\color{incolor}5}]:} \PY{n}{mytrapez}\PY{p}{(}\PY{n}{f1}\PY{p}{,}\PY{l+m+mi}{0}\PY{p}{,}\PY{n+nb}{pi}\PY{o}{/}\PY{l+m+mi}{2}\PY{p}{,}\PY{l+m+mi}{24}\PY{p}{)}
\end{Verbatim}


\begin{Verbatim}[commandchars=\\\{\}]
{\color{outcolor}Out[{\color{outcolor}5}]:} (1.2110560275684594, 6.439293542825908e-15)
\end{Verbatim}
            
    \begin{Verbatim}[commandchars=\\\{\}]
{\color{incolor}In [{\color{incolor}6}]:} \PY{n}{mySimpson}\PY{p}{(}\PY{n}{f1}\PY{p}{,}\PY{l+m+mi}{0}\PY{p}{,}\PY{n+nb}{π}\PY{o}{/}\PY{l+m+mi}{2}\PY{p}{,}\PY{l+m+mi}{4}\PY{p}{)}
\end{Verbatim}


\begin{Verbatim}[commandchars=\\\{\}]
{\color{outcolor}Out[{\color{outcolor}6}]:} (1.2114152686455435, -0.000611077902412586)
\end{Verbatim}
            
    \begin{Verbatim}[commandchars=\\\{\}]
{\color{incolor}In [{\color{incolor}7}]:} \PY{n}{mySimpson}\PY{p}{(}\PY{n}{f1}\PY{p}{,}\PY{l+m+mi}{0}\PY{p}{,}\PY{n+nb}{π}\PY{o}{/}\PY{l+m+mi}{2}\PY{p}{,}\PY{l+m+mi}{16}\PY{p}{)}
\end{Verbatim}


\begin{Verbatim}[commandchars=\\\{\}]
{\color{outcolor}Out[{\color{outcolor}7}]:} (1.2110560276465434, -9.950273232028905e-8)
\end{Verbatim}
            
    \begin{Verbatim}[commandchars=\\\{\}]
{\color{incolor}In [{\color{incolor}8}]:} \PY{n}{mySimpson}\PY{p}{(}\PY{n}{f1}\PY{p}{,}\PY{l+m+mi}{0}\PY{p}{,}\PY{n+nb}{π}\PY{o}{/}\PY{l+m+mi}{2}\PY{p}{,}\PY{l+m+mi}{24}\PY{p}{)}
\end{Verbatim}


\begin{Verbatim}[commandchars=\\\{\}]
{\color{outcolor}Out[{\color{outcolor}8}]:} (1.211056027568466, -6.556223564047059e-10)
\end{Verbatim}
            
    \hypertarget{primjer-2---pi}{%
\subsubsection{\texorpdfstring{Primjer 2 -
\(\pi\)}{Primjer 2 - \textbackslash{}pi}}\label{primjer-2---pi}}

Vrijedi

\[
\int_0^1 \frac{4}{1+x^2}\, dx=\pi.
\]

Aproksimirajmo \(\pi\) numeričkom integracijom i provjerimo pogrešku
(vidi \href{http://www.mathos.unios.hr/pim/Materijali/Num.pdf}{Numerička
matematika, poglavlje 7.3}).

Pomoću trapezne formula možemo dobiti najviše pet točnih decimala.
Simpsonova formula je točnija, ali je konvergencija spora.

    \begin{Verbatim}[commandchars=\\\{\}]
{\color{incolor}In [{\color{incolor}9}]:} \PY{n}{πbig}\PY{o}{=}\PY{k+kt}{BigFloat}\PY{p}{(}\PY{n+nb}{pi}\PY{p}{)}
\end{Verbatim}


\begin{Verbatim}[commandchars=\\\{\}]
{\color{outcolor}Out[{\color{outcolor}9}]:} 3.14159265358979323846264338327950288419716939937510582097494459
\end{Verbatim}
            
    \begin{Verbatim}[commandchars=\\\{\}]
{\color{incolor}In [{\color{incolor}10}]:} \PY{n}{f2}\PY{p}{(}\PY{n}{x}\PY{p}{)}\PY{o}{=}\PY{l+m+mi}{4}\PY{o}{/}\PY{p}{(}\PY{l+m+mi}{1}\PY{o}{+}\PY{n}{x}\PY{o}{\PYZca{}}\PY{l+m+mi}{2}\PY{p}{)}
         \PY{n+nd}{@show} \PY{n}{πapprox}\PY{o}{=}\PY{n}{mytrapez}\PY{p}{(}\PY{n}{f2}\PY{p}{,}\PY{l+m+mi}{0}\PY{p}{,}\PY{l+m+mi}{1}\PY{p}{,}\PY{l+m+mi}{10}\PY{p}{)}
         \PY{n}{πapprox}\PY{p}{[}\PY{l+m+mi}{1}\PY{p}{]}\PY{o}{\PYZhy{}}\PY{n}{πbig}
\end{Verbatim}


    \begin{Verbatim}[commandchars=\\\{\}]
πapprox = mytrapez(f2, 0, 1, 10) = (3.1399259889071587, 0.0016666250320562053)

    \end{Verbatim}

\begin{Verbatim}[commandchars=\\\{\}]
{\color{outcolor}Out[{\color{outcolor}10}]:} -1.66666468263455581230988001829434995644674312510582097494459e-03
\end{Verbatim}
            
    \begin{Verbatim}[commandchars=\\\{\}]
{\color{incolor}In [{\color{incolor}11}]:} \PY{n+nd}{@show} \PY{n}{πapprox}\PY{o}{=}\PY{n}{mytrapez}\PY{p}{(}\PY{n}{f2}\PY{p}{,}\PY{l+m+mi}{0}\PY{p}{,}\PY{l+m+mi}{1}\PY{p}{,}\PY{l+m+mi}{100}\PY{p}{)}
         \PY{n}{πapprox}\PY{p}{[}\PY{l+m+mi}{1}\PY{p}{]}\PY{o}{\PYZhy{}}\PY{n}{πbig}
\end{Verbatim}


    \begin{Verbatim}[commandchars=\\\{\}]
πapprox = mytrapez(f2, 0, 1, 100) = (3.141575986923129, 1.66666666251795e-5)

    \end{Verbatim}

\begin{Verbatim}[commandchars=\\\{\}]
{\color{outcolor}Out[{\color{outcolor}11}]:} -1.666666666423378282120949501593475335226070323082097494459e-05
\end{Verbatim}
            
    \begin{Verbatim}[commandchars=\\\{\}]
{\color{incolor}In [{\color{incolor}12}]:} \PY{n+nd}{@show} \PY{n}{πapprox}\PY{o}{=}\PY{n}{mySimpson}\PY{p}{(}\PY{n}{f2}\PY{p}{,}\PY{l+m+mi}{0}\PY{p}{,}\PY{l+m+mi}{1}\PY{p}{,}\PY{l+m+mi}{16}\PY{p}{)}
         \PY{n}{πapprox}\PY{p}{[}\PY{l+m+mi}{1}\PY{p}{]}\PY{o}{\PYZhy{}}\PY{n}{πbig}
\end{Verbatim}


    \begin{Verbatim}[commandchars=\\\{\}]
πapprox = mySimpson(f2, 0, 1, 16) = (3.141592651224822, 9.91774099882529e-9)

    \end{Verbatim}

\begin{Verbatim}[commandchars=\\\{\}]
{\color{outcolor}Out[{\color{outcolor}12}]:} -2.36497145234701707351400007371895895015635582097494459e-09
\end{Verbatim}
            
    \begin{Verbatim}[commandchars=\\\{\}]
{\color{incolor}In [{\color{incolor}13}]:} \PY{n+nd}{@show} \PY{n}{πapprox}\PY{o}{=}\PY{n}{mySimpson}\PY{p}{(}\PY{n}{f2}\PY{p}{,}\PY{l+m+mi}{0}\PY{p}{,}\PY{l+m+mi}{1}\PY{p}{,}\PY{l+m+mi}{64}\PY{p}{)}
         \PY{n}{πapprox}\PY{p}{[}\PY{l+m+mi}{1}\PY{p}{]}\PY{o}{\PYZhy{}}\PY{n}{πbig}
\end{Verbatim}


    \begin{Verbatim}[commandchars=\\\{\}]
πapprox = mySimpson(f2, 0, 1, 64) = (3.1415926535892162, 2.4253192047278085e-12)

    \end{Verbatim}

\begin{Verbatim}[commandchars=\\\{\}]
{\color{outcolor}Out[{\color{outcolor}13}]:} -5.7699434827514607372184162133321773448082097494459e-13
\end{Verbatim}
            
    \hypertarget{gaussova-kvadratura}{%
\subsection{Gaussova kvadratura}\label{gaussova-kvadratura}}

Slično kao u formuli (1), integral aproksimiramo sumom umnožaka
vrijednosti funkcije i odgovarajućih težina:

\[
\int_{a}^b \omega(x) f(x)\, dx=\sum_{k=1}^n \omega_k f(x_k),
\]

gdje je \(\omega(x)\) neka \emph{težinska funkcija}.

Točke \(x_k\) su nul-točke odgovarajućeg ortogonalnog polinoma
\(P_{n}(x)\) reda \(n+1\), na primjer, \emph{Legendreovih polinoma} za
\(\omega(x)=1\) i \emph{Čebiševljevih polinoma} za

\[\omega(x)=\displaystyle\frac{1}{\sqrt{1-x^2}}\]

(vidi bilježnicu \href{NA09\%20Interpolacijski\%20polinomi.ipynb}{NA09
Interpolacijski polinomi.ipynb}).

Težine su jednake

\[
\omega_k=\int_a^b \omega(x) \prod_{{i=1}\atop {i\neq k}}^n\frac{x-x_i}{x_k-x_i} \, dx.
\]

\emph{Pogreška} je dana s

\[
E=\frac{f^{(2n)}(\xi)}{(2n)!}\int_a^b \omega(x) P_n^2(x)\, dx.
\]

Za detalje vidi
\href{https://books.google.hr/books?id=kPDtAp3UZtIC\&hl=hr\&source=gbs_book_other_versions}{Numerical
Analysis, poglavlje 7.3}.

\textbf{Napomena}: Legendreovi i Čebiševljevi polinomi su definirani na
intervalu \([-1,1]\) pa\\
koristimo transformaciju

\[
\int_{a}^b \omega(x) f(x)\, dx = \frac{b-a}{2} \int_{-1}^1 f\bigg(\frac{b-a}{2}x+\frac{a+b}{2}\bigg) dx. \tag{3}
\]

    \hypertarget{postojeux107e-rutine}{%
\subsubsection{Postojeće rutine}\label{postojeux107e-rutine}}

Profesionalne rutine za numeričku integraciju su složene, a većina
programa ima ugrađene odgovarajuće rutine. Tako, na primjer,

\begin{itemize}
\tightlist
\item
  Ṁatlab ima rutinu \texttt{quad} koja koristi adaptivnu Simpsonovu
  formulu, a
\item
  Julia u paketu
  \href{https://github.com/JuliaMath/QuadGK.jl}{\texttt{QuadGK.jl}} ima
  rutinu \texttt{quadgk()} i računa integral u \(O(n^2)\) operacija.
\end{itemize}

Julia također ima i paket
\href{https://github.com/ajt60gaibb/FastGaussQuadrature.jl}{\texttt{FastGaussQuadrature.jl}}
koji brzo računa točke i težine za zadani \(n\) i razne težinske
funkcije pa se pomoću točaka i težina lako izračuna integral u \(O(n)\)
operacija.

    \begin{Verbatim}[commandchars=\\\{\}]
{\color{incolor}In [{\color{incolor}14}]:} \PY{k}{using} \PY{n}{QuadGK}
\end{Verbatim}


    \begin{Verbatim}[commandchars=\\\{\}]
{\color{incolor}In [{\color{incolor}15}]:} \PY{o}{?} \PY{n}{quadgk}
\end{Verbatim}


    \begin{Verbatim}[commandchars=\\\{\}]
search: \textbf{q}\textbf{u}\textbf{a}\textbf{d}\textbf{g}\textbf{k} \textbf{Q}\textbf{u}\textbf{a}\textbf{d}\textbf{G}\textbf{K}


    \end{Verbatim}
\texttt{\color{outcolor}Out[{\color{outcolor}15}]:}
    
    \begin{verbatim}
quadgk(f, a,b,c...; rtol=sqrt(eps), atol=0, maxevals=10^7, order=7, norm=norm)
\end{verbatim}
Numerically integrate the function \texttt{f(x)} from \texttt{a} to \texttt{b}, and optionally over additional intervals \texttt{b} to \texttt{c} and so on. Keyword options include a relative error tolerance \texttt{rtol} (defaults to \texttt{sqrt(eps)} in the precision of the endpoints), an absolute error tolerance \texttt{atol} (defaults to 0), a maximum number of function evaluations \texttt{maxevals} (defaults to \texttt{10\^{}7}), and the \texttt{order} of the integration rule (defaults to 7).

Returns a pair \texttt{(I,E)} of the estimated integral \texttt{I} and an estimated upper bound on the absolute error \texttt{E}. If \texttt{maxevals} is not exceeded then \texttt{E <= max(atol, rtol*norm(I))} will hold. (Note that it is useful to specify a positive \texttt{atol} in cases where \texttt{norm(I)} may be zero.)

The endpoints \texttt{a} et cetera can also be complex (in which case the integral is performed over straight-line segments in the complex plane). If the endpoints are \texttt{BigFloat}, then the integration will be performed in \texttt{BigFloat} precision as well.

\begin{quote}
\textbf{note}

Note

It is advisable to increase the integration \texttt{order} in rough proportion to the precision, for smooth integrands.

\end{quote}
More generally, the precision is set by the precision of the integration endpoints (promoted to floating-point types).

The integrand \texttt{f(x)} can return any numeric scalar, vector, or matrix type, or in fact any type supporting \texttt{+}, \texttt{-}, multiplication by real values, and a \texttt{norm} (i.e., any normed vector space). Alternatively, a different norm can be specified by passing a \texttt{norm}-like function as the \texttt{norm} keyword argument (which defaults to \texttt{norm}).

\begin{quote}
\textbf{note}

Note

Only one-dimensional integrals are provided by this function. For multi-dimensional integration (cubature), there are many different algorithms (often much better than simple nested 1d integrals) and the optimal choice tends to be very problem-dependent. See the Julia external-package listing for available algorithms for multidimensional integration or other specialized tasks (such as integrals of highly oscillatory or singular functions).

\end{quote}
The algorithm is an adaptive Gauss-Kronrod integration technique: the integral in each interval is estimated using a Kronrod rule (\texttt{2*order+1} points) and the error is estimated using an embedded Gauss rule (\texttt{order} points). The interval with the largest error is then subdivided into two intervals and the process is repeated until the desired error tolerance is achieved.

These quadrature rules work best for smooth functions within each interval, so if your function has a known discontinuity or other singularity, it is best to subdivide your interval to put the singularity at an endpoint. For example, if \texttt{f} has a discontinuity at \texttt{x=0.7} and you want to integrate from 0 to 1, you should use \texttt{quadgk(f, 0,0.7,1)} to subdivide the interval at the point of discontinuity. The integrand is never evaluated exactly at the endpoints of the intervals, so it is possible to integrate functions that diverge at the endpoints as long as the singularity is integrable (for example, a \texttt{log(x)} or \texttt{1/sqrt(x)} singularity).

For real-valued endpoints, the starting and/or ending points may be infinite. (A coordinate transformation is performed internally to map the infinite interval to a finite one.)



    

    \begin{Verbatim}[commandchars=\\\{\}]
{\color{incolor}In [{\color{incolor}16}]:} \PY{n}{quadgk}\PY{p}{(}\PY{n}{f1}\PY{p}{,}\PY{l+m+mi}{0}\PY{p}{,}\PY{n+nb}{π}\PY{o}{/}\PY{l+m+mi}{2}\PY{p}{)}
\end{Verbatim}


\begin{Verbatim}[commandchars=\\\{\}]
{\color{outcolor}Out[{\color{outcolor}16}]:} (1.2110560275684594, 8.948231045025068e-11)
\end{Verbatim}
            
    \begin{Verbatim}[commandchars=\\\{\}]
{\color{incolor}In [{\color{incolor}17}]:} \PY{n}{quadgk}\PY{p}{(}\PY{n}{f2}\PY{p}{,}\PY{l+m+mi}{0}\PY{p}{,}\PY{l+m+mi}{1}\PY{p}{)}
\end{Verbatim}


\begin{Verbatim}[commandchars=\\\{\}]
{\color{outcolor}Out[{\color{outcolor}17}]:} (3.1415926535897936, 2.6639561667707312e-9)
\end{Verbatim}
            
    \begin{Verbatim}[commandchars=\\\{\}]
{\color{incolor}In [{\color{incolor}18}]:} \PY{c}{\PYZsh{} Granice mogu biti i beskonačne}
         \PY{n}{quadgk}\PY{p}{(}\PY{n}{x}\PY{o}{\PYZhy{}}\PY{o}{\PYZgt{}}\PY{n}{exp}\PY{p}{(}\PY{o}{\PYZhy{}}\PY{n}{x}\PY{p}{)}\PY{p}{,}\PY{l+m+mi}{0}\PY{p}{,}\PY{n+nb}{Inf}\PY{p}{)}
\end{Verbatim}


\begin{Verbatim}[commandchars=\\\{\}]
{\color{outcolor}Out[{\color{outcolor}18}]:} (1.0, 4.5074000326453615e-11)
\end{Verbatim}
            
    \begin{Verbatim}[commandchars=\\\{\}]
{\color{incolor}In [{\color{incolor}19}]:} \PY{k}{using} \PY{n}{FastGaussQuadrature}
\end{Verbatim}


    \begin{Verbatim}[commandchars=\\\{\}]
{\color{incolor}In [{\color{incolor}20}]:} \PY{n}{varinfo}\PY{p}{(}\PY{n}{FastGaussQuadrature}\PY{p}{)}
\end{Verbatim}

\texttt{\color{outcolor}Out[{\color{outcolor}20}]:}
    
    \begin{tabular}
{l | r | l}
name & size & summary \\
\hline
FastGaussQuadrature & 445.386 KiB & Module \\
besselroots & 0 bytes & typeof(besselroots) \\
gausschebyshev & 0 bytes & typeof(gausschebyshev) \\
gausshermite & 0 bytes & typeof(gausshermite) \\
gaussjacobi & 0 bytes & typeof(gaussjacobi) \\
gausslaguerre & 0 bytes & typeof(gausslaguerre) \\
gausslegendre & 0 bytes & typeof(gausslegendre) \\
gausslobatto & 0 bytes & typeof(gausslobatto) \\
gaussradau & 0 bytes & typeof(gaussradau) \\
\end{tabular}


    

    \begin{Verbatim}[commandchars=\\\{\}]
{\color{incolor}In [{\color{incolor}21}]:} \PY{c}{\PYZsh{} Na primjer}
         \PY{n}{methods}\PY{p}{(}\PY{n}{gausschebyshev}\PY{p}{)}
\end{Verbatim}


\begin{Verbatim}[commandchars=\\\{\}]
{\color{outcolor}Out[{\color{outcolor}21}]:} \# 2 methods for generic function "gausschebyshev":
         [1] gausschebyshev(n::Integer) in FastGaussQuadrature at ...
         [2] gausschebyshev(n::Integer, kind::Integer) in FastGaussQuadrature at ...
\end{Verbatim}
            
    \begin{Verbatim}[commandchars=\\\{\}]
{\color{incolor}In [{\color{incolor}22}]:} \PY{n}{gausschebyshev}\PY{p}{(}\PY{l+m+mi}{16}\PY{p}{)}
\end{Verbatim}


\begin{Verbatim}[commandchars=\\\{\}]
{\color{outcolor}Out[{\color{outcolor}22}]:} ([-0.995185, -0.95694, -0.881921, -0.77301, -0.634393, -0.471397, -0.290285, 
         -0.0980171, 0.0980171, 0.290285, 0.471397, 0.634393, 0.77301, 0.881921, 
         0.95694, 0.995185], [0.19635, 0.19635, 0.19635, 0.19635, 0.19635, 0.19635, 
         0.19635, 0.19635, 0.19635, 0.19635, 0.19635, 0.19635, 0.19635, 0.19635, 
         0.19635, 0.19635])
\end{Verbatim}
            
    \begin{Verbatim}[commandchars=\\\{\}]
{\color{incolor}In [{\color{incolor}23}]:} \PY{c}{\PYZsh{} Sada računajmo integrale. U našem slučaju je ω(x)=1 }
         \PY{c}{\PYZsh{} pa nam trebaju Legendreovi polinomi}
         \PY{n}{a}\PY{o}{=}\PY{l+m+mi}{0}
         \PY{n}{b}\PY{o}{=}\PY{n+nb}{π}\PY{o}{/}\PY{l+m+mi}{2}
         \PY{n}{ξ}\PY{p}{,}\PY{n}{ω}\PY{o}{=}\PY{n}{gausslegendre}\PY{p}{(}\PY{l+m+mi}{32}\PY{p}{)}
         \PY{n}{mapnodes}\PY{p}{(}\PY{n}{x}\PY{p}{)}\PY{o}{=}\PY{p}{(}\PY{n}{b}\PY{o}{\PYZhy{}}\PY{n}{a}\PY{p}{)}\PY{o}{*}\PY{n}{x}\PY{o}{/}\PY{l+m+mi}{2} \PY{o}{.+}\PY{p}{(}\PY{n}{a}\PY{o}{+}\PY{n}{b}\PY{p}{)}\PY{o}{/}\PY{l+m+mi}{2}
\end{Verbatim}


\begin{Verbatim}[commandchars=\\\{\}]
{\color{outcolor}Out[{\color{outcolor}23}]:} mapnodes (generic function with 1 method)
\end{Verbatim}
            
    \begin{Verbatim}[commandchars=\\\{\}]
{\color{incolor}In [{\color{incolor}24}]:} \PY{c}{\PYZsh{} 1/8 opsega elipse}
         \PY{k}{using} \PY{n}{LinearAlgebra}
         \PY{p}{(}\PY{n}{b}\PY{o}{\PYZhy{}}\PY{n}{a}\PY{p}{)}\PY{o}{/}\PY{l+m+mi}{2}\PY{o}{*}\PY{n}{dot}\PY{p}{(}\PY{n}{ω}\PY{p}{,}\PY{n}{map}\PY{p}{(}\PY{n}{f1}\PY{p}{,}\PY{n}{mapnodes}\PY{p}{(}\PY{n}{ξ}\PY{p}{)}\PY{p}{)}\PY{p}{)}
\end{Verbatim}


\begin{Verbatim}[commandchars=\\\{\}]
{\color{outcolor}Out[{\color{outcolor}24}]:} 1.2110560275684594
\end{Verbatim}
            
    \begin{Verbatim}[commandchars=\\\{\}]
{\color{incolor}In [{\color{incolor}25}]:} \PY{c}{\PYZsh{} π}
         \PY{n}{a}\PY{o}{=}\PY{l+m+mi}{0}
         \PY{n}{b}\PY{o}{=}\PY{l+m+mi}{1}
         \PY{p}{(}\PY{n}{b}\PY{o}{\PYZhy{}}\PY{n}{a}\PY{p}{)}\PY{o}{/}\PY{l+m+mi}{2}\PY{o}{*}\PY{n}{dot}\PY{p}{(}\PY{n}{ω}\PY{p}{,}\PY{n}{map}\PY{p}{(}\PY{n}{f2}\PY{p}{,}\PY{n}{mapnodes}\PY{p}{(}\PY{n}{ξ}\PY{p}{)}\PY{p}{)}\PY{p}{)}
\end{Verbatim}


\begin{Verbatim}[commandchars=\\\{\}]
{\color{outcolor}Out[{\color{outcolor}25}]:} 3.141592653589793
\end{Verbatim}
            
    \hypertarget{clenshaw-curtisova-kvadratura}{%
\subsection{Clenshaw-Curtisova
kvadratura}\label{clenshaw-curtisova-kvadratura}}

Uz supstituciju \(x=\cos\theta\), vrijedi

\[
I\equiv \int\limits_{-1}^1f(x)\, dx =\int\limits_0^\pi f(\cos\theta)\sin\theta \, d\theta.
\]

Integral na desnoj strani se računa integriranjem Fourierovog reda
parnog proširenja podintegralne funkcije:

\[
I\approx a_0+\sum_{k=1}^n \frac{2a_{2k}}{1-(2k)^2},
\] pri čemu se koeficijenti \(a_k\) računaju formulom

\[
a_k=\frac{2}{\pi}\int\limits_0^\pi f(\cos\theta)\cos(k\theta)\,d\theta.
\]

Koeficijenti \(a_k\) se mogu računati numeričkom integracijom ili
korištenjem brze Fourierove transformacije (FFT), što je puno brže. Za
detalje vidi
\href{http://homerreid.dyndns.org/teaching/18.330/Notes/ClenshawCurtis.pdf}{Homer
Reid, Clenshaw-Curtis Quadrature}.

Ukoliko se integrira na intervalu \([a,b]\), prebacivanje u interval
\([-1,1]\) vrši se kao u formuli (3).

    \begin{Verbatim}[commandchars=\\\{\}]
{\color{incolor}In [{\color{incolor}26}]:} \PY{c}{\PYZsh{} Naivna implementacija}
         \PY{k}{function} \PY{n}{myClenshawCurtis}\PY{p}{(}\PY{n}{f}\PY{o}{::}\PY{k+kt}{Function}\PY{p}{,}\PY{n}{a}\PY{o}{::}\PY{k+kt}{Number}\PY{p}{,}\PY{n}{b}\PY{o}{::}\PY{k+kt}{Number}\PY{p}{,}\PY{n}{n}\PY{o}{::}\PY{k+kt}{Int64}\PY{p}{)}
             \PY{n}{mapnodes}\PY{p}{(}\PY{n}{x}\PY{p}{)}\PY{o}{=}\PY{p}{(}\PY{n}{b}\PY{o}{\PYZhy{}}\PY{n}{a}\PY{p}{)}\PY{o}{*}\PY{n}{x}\PY{o}{/}\PY{l+m+mi}{2}\PY{o}{+}\PY{p}{(}\PY{n}{a}\PY{o}{+}\PY{n}{b}\PY{p}{)}\PY{o}{/}\PY{l+m+mi}{2}
             \PY{n}{z}\PY{o}{=}\PY{k+kt}{Vector}\PY{p}{\PYZob{}}\PY{k+kt}{Float64}\PY{p}{\PYZcb{}}\PY{p}{(}\PY{n}{undef}\PY{p}{,}\PY{n}{n}\PY{p}{)}
             \PY{n}{g}\PY{p}{(}\PY{n}{x}\PY{p}{)}\PY{o}{=}\PY{n}{f}\PY{p}{(}\PY{n}{mapnodes}\PY{p}{(}\PY{n}{x}\PY{p}{)}\PY{p}{)}
             \PY{k}{for} \PY{n}{i}\PY{o}{=}\PY{l+m+mi}{1}\PY{o}{:}\PY{n}{n}
                 \PY{n}{h}\PY{p}{(}\PY{n}{x}\PY{p}{)}\PY{o}{=}\PY{n}{g}\PY{p}{(}\PY{n}{cos}\PY{p}{(}\PY{n}{x}\PY{p}{)}\PY{p}{)}\PY{o}{*}\PY{n}{cos}\PY{p}{(}\PY{l+m+mi}{2}\PY{o}{*}\PY{p}{(}\PY{n}{i}\PY{o}{\PYZhy{}}\PY{l+m+mi}{1}\PY{p}{)}\PY{o}{*}\PY{n}{x}\PY{p}{)}
                 \PY{n}{z}\PY{p}{[}\PY{n}{i}\PY{p}{]}\PY{o}{=}\PY{l+m+mi}{2}\PY{o}{*}\PY{n}{quadgk}\PY{p}{(}\PY{n}{h}\PY{p}{,}\PY{l+m+mi}{0}\PY{p}{,}\PY{n+nb}{pi}\PY{p}{)}\PY{p}{[}\PY{l+m+mi}{1}\PY{p}{]}\PY{o}{/}\PY{n+nb}{pi}
             \PY{k}{end}
             \PY{k}{return} \PY{p}{(}\PY{n}{z}\PY{p}{[}\PY{l+m+mi}{1}\PY{p}{]}\PY{o}{+}\PY{l+m+mi}{2}\PY{o}{*}\PY{n}{sum}\PY{p}{(}\PY{p}{[}\PY{n}{z}\PY{p}{[}\PY{n}{i}\PY{p}{]}\PY{o}{/}\PY{p}{(}\PY{l+m+mi}{1}\PY{o}{\PYZhy{}}\PY{l+m+mi}{4}\PY{o}{*}\PY{p}{(}\PY{n}{i}\PY{o}{\PYZhy{}}\PY{l+m+mi}{1}\PY{p}{)}\PY{o}{\PYZca{}}\PY{l+m+mi}{2}\PY{p}{)} \PY{k}{for} \PY{n}{i}\PY{o}{=}\PY{l+m+mi}{2}\PY{o}{:}\PY{n}{n}\PY{p}{]}\PY{p}{)}\PY{p}{)}\PY{o}{*}\PY{p}{(}\PY{n}{b}\PY{o}{\PYZhy{}}\PY{n}{a}\PY{p}{)}\PY{o}{/}\PY{l+m+mi}{2}
         \PY{k}{end}
\end{Verbatim}


\begin{Verbatim}[commandchars=\\\{\}]
{\color{outcolor}Out[{\color{outcolor}26}]:} myClenshawCurtis (generic function with 1 method)
\end{Verbatim}
            
    \begin{Verbatim}[commandchars=\\\{\}]
{\color{incolor}In [{\color{incolor}27}]:} \PY{c}{\PYZsh{} Implementcija pomoću trapeznog pravila s m podintervala}
         \PY{k}{function} \PY{n}{myClenshawCurtis}\PY{p}{(}\PY{n}{f}\PY{o}{::}\PY{k+kt}{Function}\PY{p}{,}\PY{n}{a}\PY{o}{::}\PY{k+kt}{Number}\PY{p}{,}\PY{n}{b}\PY{o}{::}\PY{k+kt}{Number}\PY{p}{,}\PY{n}{n}\PY{o}{::}\PY{k+kt}{Int64}\PY{p}{,}\PY{n}{m}\PY{o}{::}\PY{k+kt}{Int64}\PY{p}{)}
             \PY{n}{mapnodes}\PY{p}{(}\PY{n}{x}\PY{p}{)}\PY{o}{=}\PY{p}{(}\PY{n}{b}\PY{o}{\PYZhy{}}\PY{n}{a}\PY{p}{)}\PY{o}{*}\PY{n}{x}\PY{o}{/}\PY{l+m+mi}{2}\PY{o}{+}\PY{p}{(}\PY{n}{a}\PY{o}{+}\PY{n}{b}\PY{p}{)}\PY{o}{/}\PY{l+m+mi}{2}
             \PY{n}{z}\PY{o}{=}\PY{k+kt}{Vector}\PY{p}{\PYZob{}}\PY{k+kt}{Float64}\PY{p}{\PYZcb{}}\PY{p}{(}\PY{n}{undef}\PY{p}{,}\PY{n}{n}\PY{p}{)}
             \PY{n}{g}\PY{p}{(}\PY{n}{x}\PY{p}{)}\PY{o}{=}\PY{n}{f}\PY{p}{(}\PY{n}{mapnodes}\PY{p}{(}\PY{n}{x}\PY{p}{)}\PY{p}{)}
             \PY{n}{ls}\PY{o}{=}\PY{n}{range}\PY{p}{(}\PY{l+m+mi}{0}\PY{p}{,}\PY{n}{stop}\PY{o}{=}\PY{n+nb}{pi}\PY{p}{,}\PY{n}{length}\PY{o}{=}\PY{n}{m}\PY{o}{+}\PY{l+m+mi}{1}\PY{p}{)}
             \PY{n}{u}\PY{o}{=}\PY{n}{map}\PY{p}{(}\PY{n}{x}\PY{o}{\PYZhy{}}\PY{o}{\PYZgt{}}\PY{n}{g}\PY{p}{(}\PY{n}{cos}\PY{p}{(}\PY{n}{x}\PY{p}{)}\PY{p}{)}\PY{p}{,}\PY{n}{ls}\PY{p}{)}
             \PY{k}{for} \PY{n}{i}\PY{o}{=}\PY{l+m+mi}{1}\PY{o}{:}\PY{n}{n}
                 \PY{n}{v}\PY{o}{=}\PY{n}{map}\PY{p}{(}\PY{n}{x}\PY{o}{\PYZhy{}}\PY{o}{\PYZgt{}}\PY{n}{cos}\PY{p}{(}\PY{l+m+mi}{2}\PY{o}{*}\PY{p}{(}\PY{n}{i}\PY{o}{\PYZhy{}}\PY{l+m+mi}{1}\PY{p}{)}\PY{o}{*}\PY{n}{x}\PY{p}{)}\PY{p}{,}\PY{n}{ls}\PY{p}{)}
                 \PY{n}{z}\PY{p}{[}\PY{n}{i}\PY{p}{]}\PY{o}{=}\PY{p}{(}\PY{n}{u}\PY{p}{[}\PY{l+m+mi}{1}\PY{p}{]}\PY{o}{*}\PY{n}{v}\PY{p}{[}\PY{l+m+mi}{1}\PY{p}{]}\PY{o}{+}\PY{l+m+mi}{2}\PY{o}{*}\PY{p}{(}\PY{n}{u}\PY{p}{[}\PY{l+m+mi}{2}\PY{o}{:}\PY{k}{end}\PY{o}{\PYZhy{}}\PY{l+m+mi}{1}\PY{p}{]}\PY{o}{⋅}\PY{n}{v}\PY{p}{[}\PY{l+m+mi}{2}\PY{o}{:}\PY{k}{end}\PY{o}{\PYZhy{}}\PY{l+m+mi}{1}\PY{p}{]}\PY{p}{)}\PY{o}{+}\PY{n}{v}\PY{p}{[}\PY{k}{end}\PY{p}{]}\PY{o}{*}\PY{n}{u}\PY{p}{[}\PY{k}{end}\PY{p}{]}\PY{p}{)}\PY{o}{/}\PY{n}{m}
             \PY{k}{end}
             \PY{k}{return} \PY{p}{(}\PY{n}{z}\PY{p}{[}\PY{l+m+mi}{1}\PY{p}{]}\PY{o}{+}\PY{l+m+mi}{2}\PY{o}{*}\PY{n}{sum}\PY{p}{(}\PY{p}{[}\PY{n}{z}\PY{p}{[}\PY{n}{i}\PY{p}{]}\PY{o}{/}\PY{p}{(}\PY{l+m+mi}{1}\PY{o}{\PYZhy{}}\PY{l+m+mi}{4}\PY{o}{*}\PY{p}{(}\PY{n}{i}\PY{o}{\PYZhy{}}\PY{l+m+mi}{1}\PY{p}{)}\PY{o}{\PYZca{}}\PY{l+m+mi}{2}\PY{p}{)} \PY{k}{for} \PY{n}{i}\PY{o}{=}\PY{l+m+mi}{2}\PY{o}{:}\PY{n}{n}\PY{p}{]}\PY{p}{)}\PY{p}{)}\PY{o}{*}\PY{p}{(}\PY{n}{b}\PY{o}{\PYZhy{}}\PY{n}{a}\PY{p}{)}\PY{o}{/}\PY{l+m+mi}{2}
         \PY{k}{end}
\end{Verbatim}


\begin{Verbatim}[commandchars=\\\{\}]
{\color{outcolor}Out[{\color{outcolor}27}]:} myClenshawCurtis (generic function with 2 methods)
\end{Verbatim}
            
    \begin{Verbatim}[commandchars=\\\{\}]
{\color{incolor}In [{\color{incolor}28}]:} \PY{n}{myClenshawCurtis}\PY{p}{(}\PY{n}{f1}\PY{p}{,}\PY{l+m+mi}{0}\PY{p}{,}\PY{n+nb}{pi}\PY{o}{/}\PY{l+m+mi}{2}\PY{p}{,}\PY{l+m+mi}{8}\PY{p}{)}
\end{Verbatim}


\begin{Verbatim}[commandchars=\\\{\}]
{\color{outcolor}Out[{\color{outcolor}28}]:} 1.2110560274835651
\end{Verbatim}
            
    \begin{Verbatim}[commandchars=\\\{\}]
{\color{incolor}In [{\color{incolor}29}]:} \PY{n}{myClenshawCurtis}\PY{p}{(}\PY{n}{f1}\PY{p}{,}\PY{l+m+mi}{0}\PY{p}{,}\PY{n+nb}{pi}\PY{o}{/}\PY{l+m+mi}{2}\PY{p}{,}\PY{l+m+mi}{8}\PY{p}{,}\PY{l+m+mi}{50}\PY{p}{)}
\end{Verbatim}


\begin{Verbatim}[commandchars=\\\{\}]
{\color{outcolor}Out[{\color{outcolor}29}]:} 1.2110560274835651
\end{Verbatim}
            
    \begin{Verbatim}[commandchars=\\\{\}]
{\color{incolor}In [{\color{incolor}30}]:} \PY{n}{myClenshawCurtis}\PY{p}{(}\PY{n}{f2}\PY{p}{,}\PY{l+m+mi}{0}\PY{p}{,}\PY{l+m+mi}{1}\PY{p}{,}\PY{l+m+mi}{16}\PY{p}{,}\PY{l+m+mi}{50}\PY{p}{)}\PY{p}{,}\PY{n+nb}{pi}
\end{Verbatim}


\begin{Verbatim}[commandchars=\\\{\}]
{\color{outcolor}Out[{\color{outcolor}30}]:} (3.1415926535897936, π = 3.1415926535897{\ldots})
\end{Verbatim}
            
    \begin{Verbatim}[commandchars=\\\{\}]
{\color{incolor}In [{\color{incolor}31}]:} \PY{n}{myClenshawCurtis}\PY{p}{(}\PY{n}{x}\PY{o}{\PYZhy{}}\PY{o}{\PYZgt{}}\PY{n}{exp}\PY{p}{(}\PY{o}{\PYZhy{}}\PY{n}{x}\PY{p}{)}\PY{p}{,}\PY{l+m+mi}{0}\PY{p}{,}\PY{l+m+mi}{1000}\PY{p}{,}\PY{l+m+mi}{100}\PY{p}{,}\PY{l+m+mi}{200}\PY{p}{)}
\end{Verbatim}


\begin{Verbatim}[commandchars=\\\{\}]
{\color{outcolor}Out[{\color{outcolor}31}]:} 1.0000000000000009
\end{Verbatim}
            
    \begin{Verbatim}[commandchars=\\\{\}]
{\color{incolor}In [{\color{incolor}32}]:} \PY{c}{\PYZsh{} Najbrža implementacija pomoću fft(), 2\PYZca{}n je broj točaka}
         \PY{k}{using} \PY{n}{FFTW}
         \PY{k}{function} \PY{n}{myClenshawCurtis}\PY{p}{(}\PY{n}{f}\PY{o}{::}\PY{k+kt}{Function}\PY{p}{,}\PY{n}{a}\PY{o}{::}\PY{k+kt}{Number}\PY{p}{,}\PY{n}{b}\PY{o}{::}\PY{k+kt}{Number}\PY{p}{,}\PY{n}{n}\PY{o}{::}\PY{k+kt}{Int64}\PY{p}{)}
             \PY{n}{mapnodes}\PY{p}{(}\PY{n}{x}\PY{p}{)}\PY{o}{=}\PY{p}{(}\PY{n}{b}\PY{o}{\PYZhy{}}\PY{n}{a}\PY{p}{)}\PY{o}{*}\PY{n}{x}\PY{o}{/}\PY{l+m+mi}{2}\PY{o}{+}\PY{p}{(}\PY{n}{a}\PY{o}{+}\PY{n}{b}\PY{p}{)}\PY{o}{/}\PY{l+m+mi}{2}
             \PY{n}{g}\PY{p}{(}\PY{n}{x}\PY{p}{)}\PY{o}{=}\PY{n}{f}\PY{p}{(}\PY{n}{mapnodes}\PY{p}{(}\PY{n}{x}\PY{p}{)}\PY{p}{)}
             \PY{n}{w}\PY{o}{=}\PY{n}{map}\PY{p}{(}\PY{n}{x}\PY{o}{\PYZhy{}}\PY{o}{\PYZgt{}}\PY{n}{g}\PY{p}{(}\PY{n}{cos}\PY{p}{(}\PY{n}{x}\PY{p}{)}\PY{p}{)}\PY{p}{,}\PY{n}{range}\PY{p}{(}\PY{l+m+mi}{0}\PY{p}{,}\PY{n}{stop}\PY{o}{=}\PY{l+m+mi}{2}\PY{o}{*}\PY{n+nb}{pi}\PY{p}{,}\PY{n}{length}\PY{o}{=}\PY{l+m+mi}{2}\PY{o}{\PYZca{}}\PY{n}{n}\PY{p}{)}\PY{p}{)}
             \PY{n}{w}\PY{p}{[}\PY{l+m+mi}{1}\PY{p}{]}\PY{o}{=}\PY{p}{(}\PY{n}{w}\PY{p}{[}\PY{l+m+mi}{1}\PY{p}{]}\PY{o}{+}\PY{n}{w}\PY{p}{[}\PY{k}{end}\PY{p}{]}\PY{p}{)}\PY{o}{/}\PY{l+m+mi}{2}
             \PY{n}{z}\PY{o}{=}\PY{n}{real}\PY{p}{(}\PY{n}{fft}\PY{p}{(}\PY{n}{w}\PY{p}{)}\PY{p}{)}
             \PY{n}{z}\PY{o}{/=}\PY{l+m+mi}{2}\PY{o}{\PYZca{}}\PY{p}{(}\PY{n}{n}\PY{o}{\PYZhy{}}\PY{l+m+mi}{1}\PY{p}{)}
             \PY{k}{return} \PY{p}{(}\PY{n}{z}\PY{p}{[}\PY{l+m+mi}{1}\PY{p}{]}\PY{o}{+}\PY{l+m+mi}{2}\PY{o}{*}\PY{n}{sum}\PY{p}{(}\PY{p}{[}\PY{n}{z}\PY{p}{[}\PY{n}{i}\PY{p}{]}\PY{o}{/}\PY{p}{(}\PY{l+m+mi}{1}\PY{o}{\PYZhy{}}\PY{p}{(}\PY{n}{i}\PY{o}{\PYZhy{}}\PY{l+m+mi}{1}\PY{p}{)}\PY{o}{\PYZca{}}\PY{l+m+mi}{2}\PY{p}{)} \PY{k}{for} \PY{n}{i}\PY{o}{=}\PY{l+m+mi}{3}\PY{o}{:}\PY{l+m+mi}{2}\PY{o}{:}\PY{l+m+mi}{2}\PY{o}{\PYZca{}}\PY{p}{(}\PY{n}{n}\PY{o}{\PYZhy{}}\PY{l+m+mi}{1}\PY{p}{)}\PY{p}{]}\PY{p}{)}\PY{p}{)}\PY{o}{*}\PY{p}{(}\PY{n}{b}\PY{o}{\PYZhy{}}\PY{n}{a}\PY{p}{)}\PY{o}{/}\PY{l+m+mi}{2}
         \PY{k}{end}
\end{Verbatim}


\begin{Verbatim}[commandchars=\\\{\}]
{\color{outcolor}Out[{\color{outcolor}32}]:} myClenshawCurtis (generic function with 2 methods)
\end{Verbatim}
            
    \begin{Verbatim}[commandchars=\\\{\}]
{\color{incolor}In [{\color{incolor}33}]:} \PY{n}{myClenshawCurtis}\PY{p}{(}\PY{n}{f1}\PY{p}{,}\PY{l+m+mi}{0}\PY{p}{,}\PY{n+nb}{pi}\PY{o}{/}\PY{l+m+mi}{2}\PY{p}{,}\PY{l+m+mi}{11}\PY{p}{)}
\end{Verbatim}


\begin{Verbatim}[commandchars=\\\{\}]
{\color{outcolor}Out[{\color{outcolor}33}]:} 1.2113091816227997
\end{Verbatim}
            
    \begin{Verbatim}[commandchars=\\\{\}]
{\color{incolor}In [{\color{incolor}34}]:} \PY{n}{myClenshawCurtis}\PY{p}{(}\PY{n}{f2}\PY{p}{,}\PY{l+m+mi}{0}\PY{p}{,}\PY{l+m+mi}{1}\PY{p}{,}\PY{l+m+mi}{16}\PY{p}{)}\PY{p}{,}\PY{n+nb}{pi}
\end{Verbatim}


\begin{Verbatim}[commandchars=\\\{\}]
{\color{outcolor}Out[{\color{outcolor}34}]:} (3.141571198232438, π = 3.1415926535897{\ldots})
\end{Verbatim}
            

    % Add a bibliography block to the postdoc
    
    
    
    \end{document}
